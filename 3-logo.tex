\chapter{Logo} \label{cap:papert2}

This chapter is an introduction to the use of Logo. It is addressed mainly to teacher students, trying to show how to initiate kids to the Logo practice. Interestingly, having proposed this matter to several hundreds of teacher students,  I realized how teaching Logo to kids is very similar to teaching it to grown up students. And, in average, during the first approach kids perform even better! We often discuss it with those students that claim having had major difficulties talking to the Turtle. The majority of them experience a kind of "resurrection" after some initial struggling: "At the beginning I didn't understand anything but then... what a marvel!"  But few never free themselves from the bad mood, at least within the two and a half weeks duration of the lab. Talking with these students is always very interesting because they are amazed when I tell them that, most of the times, 9 year old kids get along with the Turtle quite well. When trying to dig into this paradox, it appears that kids grasp quite naturally the playful aspect of the situation. On the other side, as we have pointed out in the previous chapter, adults easily get trapped in their own prejudices: I don't have a head for numbers, computer and me are very far away, I never went along with technologies. Especially the last one is a statement I heard lots of times, and, amazingly, from true digital natives. That is, a person who may have thousands of Facebook contacts, candidly claims of not getting along with technologies.

Having stressed that, please, do face the Turtle with the most playful spirit possible, starting from scratch. Since I suppose you know how kids tackle a new game, try doing just the same. Having stressed that, please, do face the Turtle with the most playful spirit possible, starting from scratch. Since I suppose you know how kids tackle a new game, try doing just the same.  Now, provided you downloaded LibreoOffice and activated the Logo toolbar, as explained in the first chapter, open a new text document. You are in front of a white sheet: you could write a letter, a shopping list, a poem or whatever you like. Try to write the following\footnote{I used uppercase characters just for clarity, LibreLogo is "case insensitive"}:

\vskip 1cm

\begin{scriptsize}
\begin{minipage}{0.45\textwidth}
\begin{itemize}[itemsep=-3pt,parsep=2pt]
\item[] \hspace{0.5cm} FORWARD 100
\end{itemize}
\end{minipage}
\end{scriptsize}

\vskip 1cm

What do you see?  Forget your "limits" and everything you think about your relationships with math, technology and so on. Look at the result you have obtained and compare it with what you have written. Ask yourself questions. Experiment with the Turtle about the possible answers that come in your mind. Before trying something new, write the following commands before the previous one:

\vskip 1cm

\begin{scriptsize}
\begin{minipage}{0.45\textwidth}
\begin{itemize}[itemsep=-3pt,parsep=2pt]
\item[] \hspace{0.5cm} CLEARSCREEN 
\item[] \hspace{0.5cm} HOME
\end{itemize}
\end{minipage}
\end{scriptsize}

\vskip 1cm

With the first one you cancel the previous drawing, with the second one you send the Turtle in its "home" position, that is at the center of the sheet with its nose pointing upside.

\vskip 1cm

\begin{scriptsize}
\begin{minipage}{0.45\textwidth}
\begin{itemize}[itemsep=-3pt,parsep=2pt]
\item[] \hspace{0.5cm} CLEARSCREEN 
\item[] \hspace{0.5cm} HOME
\item[] \hspace{0.5cm} FORWARD 50
\end{itemize}
\end{minipage}
\end{scriptsize}

\vskip 1cm

Then, again, press the "Run" button. What's changed? 

Now, in order to explore better what the Turtle can do, I tell you a couple of other commands: RIGHT and LEFT. Not difficult to imagine what are these commands for, isn't it? But something is missing, in order to be able to use it, what? Well, try to guess and experiment. Please, reflect on what's really going on with the Turtle, when you apply these commands. Then, just play to see what you can do. Remember, as you were I child...

The reason for I'm insisting so much about what children can do is that I had several opportunities to work with them and with the Turtle. really a few say it's too difficult, most just play and get excited to let the small animal create funny things. Once, trying to draw a house a girl came out with kind of a bizare castle. I praised her, commenting on how marvelous was this drawing, she was so proud. Then, when I came back to her, having considered the works of other kids, I found her in tears, sobbing desperately: she had accidentaly erased the commands in a way that I have been no more able to recover. I felt guilty for not having taught her how to save the work once in a while.

Therefore, if you start creating some more complex stuff you like, don't forget to save the document once in a while. It's so easy.

In a past version of this book this chapter was much longer. Following the experience I have gathered during the last couples of years, both working with teacher students and with kids, I prefer to stop here, inviting you to freely explore the possibilities. From here you can go in several directions. If you want specific indications and examples for using the basic commands, or you want to discover more powerful instructions, you'll find them in the next chapter and following ones.  In chapter \ref{parte:esperienze-didattiche} you can discover the possibilities of explorations by reading the story of Marta, a former teacher student of mine. Or, if you want to read about a fascinating way about introducing kids to the drawing of a circle, go and read chapter \ref{cap:cerchio}, well, the first part of it, otherwise you'll begin flying...
 

%Turtle geometry is a true geometry, like that of Euclid or Descartes; Euclid's is logical, Descartes's is algebraic whereas Turtle geometry is computational. In Euclidean geometry the point is the entity with just a position and  nothing else, no size, no shape. For people poorly familiar with mathematics the concept of Euclidean point may sound weird, difficult to relate to a real life object. In Turtle geometry the role of the Euclidean point is played by the Turtle. The Turtle is a much more intuitive image, even if soon we will have to relate it with a position - which we could identify with her nose tip, for instance, but this is irrelevant. However, there is something more: a point has just its position, the Turtle has a position but also a "heading". You can identify yourself with the Turtle. This gives to kids the possibility of relating formal mathematical concepts to their physical experience. Let's see how this may happen.

%The Turtle understands some specific commands expressed in a language that Papert called "Turtle talk". The command FORWARD causes the Turtle to move in a straight line along the direction it is facing. To tell it the distance to travel you have to add a number after the command: FORWARD 1  will cause a small movement, FORWARD 100 a larger one. Often kids are initiated to this logic through a mechanical Turtle, like the Bee-Bot, which represents a bee, actually, but the concept is the same. This is a simple robot that can be programmed by means of some keys located on its back, reproducing the basic commands of Turtle talk \footnote{There is also the Blue-Bot, which is a little bit more complex, having the capability to receive a sequence of commands in a row from a smartphone app via a bluetooth connection}. It's worthwhile to remember that such a cybernetic version of the Turtle was the first one created by Papert in the seventies (Fig. 1). 

%When children get involved in both activities, with the Bee-Bot (or similar) and with Logo, they experience something that has to do with a very powerful mathematical concept, Papert would say a powerful idea, that of a "isomorphism": a relationship involving a biunivocal correspondence between two completely different worlds.

%The FORWARD and BACK commands cause the Turtle to move straightforward along its haeding: the position changes but the heading remains the same. There are also commands that change the heading without affecting the position: RIGHT and LEFT. By means of these commands the Turtle pivots without changing its position. In order to work, like FORWARD, they need a number. For an adult it is easy to recognize in these numbers the turning angle in degrees. However children have to explore them and often they do this with a lot of fun.



