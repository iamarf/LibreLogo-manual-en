\chapter{The fear of math} \label{cap:papert}

The fundamental motivation for the genesis of Logo lies in the still unresolved question of mathematical teaching. The Logo language was conceived by Papert precisely to try to solve this age-old problem for which he had also coined a precise name, \textit{Mathofobia}, to describe the widespread antipathy towards this subject. Papert's interest in this issue has accompanied his entire working life, which  extended along the second half of the 20th century. His contribution was exceptional, both in terms of theoretical elaboration from the pedagogical point of view, and of the creativity that led him to devise a language specifically to bring children closer to mathematics. His profound competence in both mathematical-informatics and pedagogy makes his work unique and explains his rare ability to propose concrete solutions.

The feeling is that not much has changed, since the 1980s, at least on average\footnote{This statement hides a world of perplexity. What has changed? Perhaps that is what a mere average does not express. The school to which Papert referred is probably more similar to the one I attended (I grade in 1960). At that time, the panorama was probably much more even. "Beat him if he doesn't understand why he's a goose!" recommended the mother of a classmate of mine to the teacher. Parents were allies of that school system, in a educational vision that could be coercive and punitive, but that ran through all kinds of schools and all social strata. There were no "parent coaches" or "unionist parents". Families worked hard, school was hard. There wasn't yet any "free time". The school was more brutal, perhaps unfair, the pedagogy simple, but the picture was clearer. At least in the rural province of the 60s where I lived. Nowadays complexity reigns supreme. The categories intersect. The debates explode, amplified by the media, at a microscopic level (groups of parents in Wathsapp or Facebook) and at a macroscopic level (press, television, etc.). My personal experiences are schizophrenic: my contacts with the world of teaching represent a fascinating picture of commitment, study and experimentation; but private stories and the stories of acquaintances are populated by obsolete and superficial didactic practices. Variability is amazing. Where is the average? Frankly, I cannot assess it, but the dispersion is certainly much wider than it once was. The picture is complicated by international investigations, marked by scientific rigour but that may turn out to be fatuous. For some years, Finlandâs polar star has shone in the sky of OECD PISA assessments, particularly about mathematics. But at the same time you can stumble upon a number of complaints from Finnish academics about a collapse in mathematical skills: it seems that Finnish students have become good at PISA mathematical testing but have worsened in mathematics! Reading Giorgio Israel's post "The Bluff of Finnish Mathematics" (http://gisrael.blogspot.it/2011/05/il-bluff-della-matematica-finlandese.html), which summarizes these complaints, we discover that learning models are trivially utilitarian and far from Papert's ideas. Where will the truth be? In short, confusion reigns supreme and one seriously wonders whether one should not resign oneself to considering it just inevitable.}; that the initial motivation, based on a serious and difficult revisitation of the way young people are introduced to math, has ended up diluted in the pot of "coding", sort of Disneyland, superficially exciting for some, object of derision for others; that Papert's message, in some ways extreme and provocative, certainly to be decoded with respect to a changed context, is largely mistaken; that everything is defeated by the failure of Logo, restricted to a minority of experimental circles, without having revolutionized anything, in contrast with  Papert's legitimate expectations; that invoking the magic of mathematics to introduce young people to a domain commonly considered "cold", is a dream conceivable only by a mathematician, somewhat 'idealistic. In other words an utopia.

Logo failed we said. It failed in Papert's initial intentions. It is not widespread in the schools and it is not adopted as a standard way to support math and science learning. But it did not fail in the sense of not having left a trace, quite the contrary. There are many versions of Logo around the world, some of which became important tools for educational investigation, for example in the field of simulation of complex biological systems. And it is always from Logo that the sprawling world of visual block languages took its cue, first and foremost Scratch. Logo and Scratch are not in opposition. In a way, Scratch derives from Logo and "contains" many of its functionalities. Many of the things you can do in Logo can also be done in Scratch. But Scratch is much more oriented to the building of "you own videogame" or to the storytelling. The problem, however, is that this wider range of possibilities, deployed in a context of poorly technological educated teachers, has ended up dispersing the original educational intentions of Logo. One of the intentions of this manual is to recover the original "mathematical flavor" of the coding practice at school. In the next section we are going to comment what Papert called "Mathophobia", a kind of illness that Logo was intended to fight.

\section{\textit{\textit{Mathophobia}: The Fear for Learning}}

Seymour Papert chose to title the second chapter of Mindstorms "Mathophobia: The Fear for Learning" . Its starting point is the schizophrenic split between humanities and science, a division that is deeply built in the language, the worldview, the social organization and the educational system. A division that in the last 30-40 years of neoliberistic drift, has further widened. For instance, in the universities , since the 1980s, academics struggle in the competition for getting their researches funded. Unfortunately, the other side of the coin is that almost nobody cares about teaching, or very little. The career of a professor does not depend on the quality of his teaching but almost only on the quantity of her or his scientific outputs. The consequences are too bad. Academics tend to transform in managers when they do research and public civil servant when they teach: dynamic entrepreneurs on one hand and (strong) conservatives the other. In that way, even in the humanities we see a technical drift of the academic role. The mission of teaching, which in a sense is the "humanistic side"  of the job  is reduced to a kind of Cinderella. Thus, the dichotomization between scientific and humanistic is even stronger and unbalanced.

The issue is not that of some proper balance but to break the line between the two cultures. Papert looked at the computer as a force to dampen the distinction. The practice of coding was thought as a way to introduce a more humanistic mathematics and to exploit some scientific reasoning in the humanities. It is in this context that Papert talked about a \textit{Mathland}, where mathematics would become a natural vocabulary, with the idea that we could change not only how we teach math but even the way in which our culture thinks about knowledge and learning.

Papert claims that his arguments are not limited to the  learning of math but concern the attitude to learning in general. The word \textit{mathophobia} suggests two associations. One is the widespread dislike of mathemathics. The other derives from the stem "math", that in ancient greek means learning in general sense. Thus, if children begin by being skillful spontaneous learners, later on they "learn" the fear of learning and not only of mathematics. Ironically, it seems that the more you get instructed the more you fear learning.

Children learn thousands of words before entering the first grade. Less obvious to many people is the fact that kids learn a great deal of mathematics as well. Among  these preschool knowledges there are for instance notions such as the volume conservation of liquids in vessels of different shape, or the independence of the total number of objects from the order in which they have been counted. Papert called this

\begin{quote}Piagetian learning, a learning process that has many features the schools should envy: it is effective (all the children get there), it is inexpensive (it seems to require neither teachers nor a curriculum development), and it is humane (the children seem to do it in a carefree spirit without explicit external rewards and punishments).
\end{quote}.

As a matter of fact, the practices of mathematics teaching largely underestimates the Piagetian learning, imposing formal knowledges that turn out to be mostly dissociated from the former spontaneous notions. The consequences for the future adults are heavy. The loss of the child's positive attitude towards learning is a very common phenomenon of adult ages and it does not concern only mathematics.

\begin{quote}
Deficiency becomes identity: "I can't learn French, I don't have an ear for languages;" "I could never be a businessman, I don't have a head for figures."
\end{quote}.  

Some 80\% of my Primary Education students declare themselves as "distant from math". Many claim to have difficulties with technologies as well, despite they are supposed to be "digital natives".

The notion that there are smart people and dumb people for a given activity is widespread. It is extremely difficult to eradicate the prejudices about one's own attitudes. The point is that the acepted beliefs about mathematical aptitude do not follow from the available evidence. In order to reinforce the concept Papert recasted the argument as follows \cite{Papert} (p. 43):

\begin{quote}
Imagine that children were forced to spend an hour a day drawing dance steps on the squared paper and had to pass tests in these “dance facts” before they were allowed to dance physically. Would we not expect the world to be full  full of “dancophobes” ? Would we say that those who made it to the dance floor and music had the greatest “aptitude for dance”? In my view it is no more appropriate to draw conclusions about mathematical aptitude from children's unwillingness to spend many hundreds of hours doing sums.
\end{quote}.  

School constructs aptitudes:

\begin{quote}
Consider the case of a child I observed through his 8th and 9th years. Jim was a highly verbal and mathophobic child from a professional family. His love for words and for talking showed itself very early, long before he went to school. The mathophobia developed at the school. My theory is that it came as a direct result of his verbal precocity. I learned from his parents the Jim had developed an early habit of describing in words, often aloud, whatever he was doing as he did it. This habit caused him minor difficulties with parents and preschool teachers. The real trouble came when he hit the arithmetic class. By this time he had learned to keep "talking aloud" under control, but I believe that he still maintained his inner running commentary on his activities. In his math class he was stymied: he simply did not know how to talk about doing sums. He lacked a vocabulary (as most of us do) and a sense of purpose. Out of this frustration of his verbal habits grew a hatred of math, and out of the hatred grew what the tests later confirmed as poor attitude.

For me the story is poignant. I am convinced that what shows up as intellectual weakness very often grows, as Jim's did, out of intellectual strengths. And it is not only verbal strengths that undermine others. Every careful observer of children must have seen similar processes working in different directions: for example, a child who has become enamored of logical order is set up to be turned off by English spelling and to go on from there to develop a global dislike for writing.  
\end{quote}.  

Papert's idea was that we could use computers as vehicles to escape from the situation of Jim or that of children loving logic but with kind of dyslexic problems. In both cases they are victims of our culture's sharp separation between the verbal and the mathematical. He imagines a \textit{Mathland} where Jim's love and skill for language could be mobilized to serve his formal mathematical learning instead of opposing it, whereas for the other kind of children, the love for logic could nourish the interest in linguistics. The prevailing teaching methods give mathematics learners limited possibilities to make sense of what they are learning. Consequently, children are forced to follow the worst model for learning mathematics, which is rote learning, where material is meaningless. It is what Papert called a \textit{dissociated} model.

\begin{quote}
Well into a year-long study that put powerful computers in the classrooms of a group of “average” 7th graders, the students were at work on what they called “computer poetry”.  They were using computer programs to generate sentences. They gave the computer a syntactic structure within which to make random choices from given lists of words. The result is the kind of concrete poetry we see in the illustration that follows\footnote{[NdR] I reported this example because we will find it again, in different form, among the more advanced applications of Logo. Interestingly, in the 70s you needed a powerful computer and an advanced research staff, nowadays you can do the same thing with a simple implementation of Logo in a PC.}. One of the students, a 13 year old named Jenny, had deeply touched the project’s staff by asking on the first day of her computer work, “Why where we chosen for this? We’re not the brains”. The study had deliberately chosen children of “average” school performance. One day Jenny came in very excited. She had made a Discovery. “ Now I know why we have nouns and verbs,” she said.  For many years in school Jenny had been drilled in grammatical categories. She had never understood the differences between nouns and verbs and adverbs.  But now it was apparent that the difficulty with grammar was not due to an inability to work with logical categories. It was something else. She had simply seen no purpose in the enterprise. She had not been able to make any sense of what grammar was about in the sense of what it might be \textit{for}. And when she had asked what it was for, the explanations that her teachers gave seemed manifestly dishonest. She said she had been told that “grammar helps you talk better.
\end{quote}

\begin{quote}
INSANE RETARD MAKES BECAUSE SWEET SNOOPY SCREAMS\\
SEXY GIRL LOVES THATS WHY THE SEXY LADY HATES\\
UGLY MAN LOVES BECAUSE UGLY DOG HATES\\
MAD WOLF HATES BECAUSE INSANE WOLF SKIPS\\
SEXY RETARD SCREAMS THATS WHY THE SEXY RETARD\\
THIN SNOOPY RUNS BECAUSE FAT WOLF HOPS\\
SWEET FOGINY SKIPS A FAT LADY RUNS\\
\end{quote}

\begin{quote}
In fact, tracing the connection between learning grammar and improving speech requires a more distanced view of the complex process of learning language then Jenny could have been given at the age she first encountered grammar. She certainly didn't see any way in which grammar could help talking, nor did she think her talking needed any help full stop  therefore she learnt to approach grammar with resentment. And, as is the case for most of us, resentment guaranteed value. But now, as she tried to get the computer to generate poetry, something remarkable happened.  She found herself classifying words into categories not because she had been told she had to but because she needed to. In order to teach a computer to make strings of words that would look like English, she had to teach it to choose words of an appropriate class. What she learnt about grammar from this experience with a machine was anything but mechanical or routine. How learning was deep and meaningful.  Jenny did more than learn definitions for particular grammatical classes. She understood the general idea that words (like things) can be placed in different groups or sets, and that doing so could work for her. She not only “understood” grammar, she changed her relationship to it. It was hers, and during her year with the computer, incidents like these helped Jenny change her image of herself.  Her performance changed too; her previously low to average grades became “straight A's” for her remaining years of school. She learned that she could be a brain after all.
\end{quote}

Papert put his argument in a very strong way: often children cannot understand what are math and grammar for because they perceive adult explanations as double talk.

\begin{quote}
It is easy to understand why math and grammar seem to make sense to children when they fail to make sense to everyone around them and why helping children to make sense of them requires more than a teacher making the right speech or putting the right diagram on the board. I have asked many teachers and parents what they thought mathematics to be and why it was important to learn it. Few held a view of mathematics that was sufficiently coherent to justify devoting several thousand hours of a child's life to learning it, and children sense this. When a teacher tells a student that the reason for those many hours of arithmetic is to be able to check the change at the supermarket, the teacher is simply not believed. Children see such reasons as one more example of adult double talk. The same effect is produced when children are told school math is “fun” when they are pretty sure that teachers who say so spend their leisure hours on anything except this allegedly fun-filled activity. Nor does it help to tell them that they need math to become scientists - most children don't have such a plan. The children can see perfectly well that the teacher does not like math anymore than they do and that the reason for doing it is simply that it has been inscribed into the curriculum.  All of these erodes children's confidence in the adult world and the process of education.  \textit{And I think it introduces a deep element of dishonesty into the educational relationship}.
\end{quote}

It is important to keep in mind the difference between mathematics - a vast domain of inquiry whose beauty is rarely suspected by most laymen - and \textit{school math}. The latter is a kind of social construction, that is a set of mathematical topics determined by a succession of specific circumstances. A process that do not guarantees, \textit{per se}, the achievement of an optimal result. It reminds the story of the QWERTY keyboard layout. QWERTY represents the first five keys in the upper rows. This arrangement has no rational explanation but only an historical one. It was introduced because the keys of the first typewriters tended to jam. So they were arranged to reduce collisions by separating the keys that followed one another most frequently. The technology of typewriters improved rapidly and in a few years the jamming problem was no more an issue but the QWERTY arrangement stuck. A this point too many people were fluent with the QWERTY layout and the production of typewriters was too far away to make a step back for redesigning a more rational layout, for instance by grouping the most used keys together. 

The QWERTY problem is a good example of how consolidated habits may not necessarily be the best choice. Like the QWERTY layout, school math was shaped in a different historical context. In the same way, this idea of mathematics has dug itself deeply and, even nowadays, for most people it is inconceivable that math could be also something else. I remember a well known professor of calculus who, at the beginning of the first year of the mathematics curriculum, exhorted his students to forget what they had learned in high school since math was something different. 

Turtle geometry was conceived to fit children and, first of all, to be \textit{appropriable}. We could describe this concept by means of some principles. First, the \textit{continuity principle}: new mathematical knowledge has to be continuous with the existing one, the one kids have before going to school. Then the \textit{power principle}: new knowledge must empower learners to realize personally meaningful projects, that could not be done without it. Finally, the \textit{principle of cultural resonance}: new concepts must make sense to kids in their social context. Ironically, even in adults social context: we should not inflict on children something we have not thoroughly understood and, unfortunately, with "classic basic math" this is the case. 


