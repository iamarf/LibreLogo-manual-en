\chapter{Appendice} \label{cap:appendice}

Qui raccogliamo i listati dei codici usati per costruire alcune delle figure usate nel manuale. Naturalmente, questi codici talvolta possono contenere dei costrutti che non sono stati ancora affrontati nel punto del testo cui appare la figura corrispondente. Non importa se non si capisce tutto subito, il lettore può sempre tornarci successivamente, quando avrà sarà più esperto. Li mettiamo a disposizione perché può essere interessante e utile per vedere che cosa si può fare in pratica.

I listati sono commentati. In LibreLogo i commenti si ottengono preponendo un punto e virgola: in qualsiasi riga, tutto quello che segue il punto e virgola non viene eseguito ma serve solo a rendere più facile da leggere il codice. Questo significa che se salviamo il codice di una di queste figure, così com'è, con tutti commenti, questo può essere seguito per produrre la figura. 

È molto importante inserire nel codice codice commenti chiari e accurati, sia per rileggerlo più facilmente molto tempo dopo, sia per facilitare la collaborazione con altre persone. Scrivendo codice, specie quando si è acquisita una certa confidenza, è facile farsi prendere la mano, cercando di arrivare quanto più velocemente al risultato desiderato. È bene invece controllarsi, imponendosi di documentare adeguatamente i lavoro man mano che si procede. Tanto più si aspetta quanto più sarà faticoso andare a commentare il lavoro fatto, sia per la mole che per la maggiore difficoltà a ricordare i particolari. Naturalmente questo vale per il codice destinato durare un certo tempo e ad essere condiviso, come potrebbe essere il caso di quello scritto per costruire alcune delle figure di questo manuale. Non certo per piccoli frammenti estemporanei. Il codice riportato nelle seguenti pagine è un po' al limite. Ma lo abbiamo commentato per mostrare la buona pratica, ivi inclusa l'apposizione di un'intestazione che riporti anche il nome dell'autore, il numero della versione e la data. Intestazione che andrà aggiornata con eventuali successive modifiche.


\vskip 1cm

%https://en.wikibooks.org/wiki/LaTeX/Source_Code_Listings
\lstset{literate=
  {á}{{\'a}}1 {é}{{\'e}}1 {í}{{\'i}}1 {ó}{{\'o}}1 {ú}{{\'u}}1
  {Á}{{\'A}}1 {É}{{\'E}}1 {Í}{{\'I}}1 {Ó}{{\'O}}1 {Ú}{{\'U}}1
  {à}{{\`a}}1 {è}{{\`e}}1 {ì}{{\`i}}1 {ò}{{\`o}}1 {ù}{{\`u}}1
  {À}{{\`A}}1 {È}{{\'E}}1 {Ì}{{\`I}}1 {Ò}{{\`O}}1 {Ù}{{\`U}}1
  {ä}{{\"a}}1 {ë}{{\"e}}1 {ï}{{\"i}}1 {ö}{{\"o}}1 {ü}{{\"u}}1
  {Ä}{{\"A}}1 {Ë}{{\"E}}1 {Ï}{{\"I}}1 {Ö}{{\"O}}1 {Ü}{{\"U}}1
  {â}{{\^a}}1 {ê}{{\^e}}1 {î}{{\^i}}1 {ô}{{\^o}}1 {û}{{\^u}}1
  {Â}{{\^A}}1 {Ê}{{\^E}}1 {Î}{{\^I}}1 {Ô}{{\^O}}1 {Û}{{\^U}}1
  {œ}{{\oe}}1 {Œ}{{\OE}}1 {æ}{{\ae}}1 {Æ}{{\AE}}1 {ß}{{\ss}}1
  {ű}{{\H{u}}}1 {Ű}{{\H{U}}}1 {ő}{{\H{o}}}1 {Ő}{{\H{O}}}1
  {ç}{{\c c}}1 {Ç}{{\c C}}1 {ø}{{\o}}1 {å}{{\r a}}1 {Å}{{\r A}}1
  {€}{{\euro}}1 {£}{{\pounds}}1 {«}{{\guillemotleft}}1
  {»}{{\guillemotright}}1 {ñ}{{\~n}}1 {Ñ}{{\~N}}1 {¿}{{?`}}1
}
\lstset{extendedchars=true, basicstyle=\scriptsize} 
\begin{lstlisting}[frame=single]  % Start your code-block

; FIGURA 1 (torna alla figura)
; Le coordinate della patorna alla figuragina
; Versione 1.
; A.R. Formiconi
; 25 luglio 2016


; Versione scritta come viene, senza particolari ottimizzazioni
                                                                
;  Predispone tutto

CLEARSCREEN
HOME
A = 200				; lato corto del rettangolo che 
                                ; rappresenta la pagina
B = 282				; lato lungo
FILLCOLOR [230, 230, 230]	; fissa a grigio chiaro il colore 
                                ; di riempimento
PENUP
FORWARD B/2			; si dirige verso l'angolo in alto 
                                ; a sinistra
HEADING 9h
FORWARD A/2
HEADING 3h			; e punta verso destra (ore 3)
PENDOWN

; disegna il rettangolo, ruotando in senso orario

FORWARD A
RIGHT 90
FORWARD B
RIGHT  90
FORWARD A
FILL				; riempie il rettangolo
PENUP
RIGHT 90
FORWARD B			; si posiziona nell'angolo in alto a 
                                ; sinistra del foglio

; ripete il giro marcando gli angoli e ponendo le scritte 

HEADING 3h			; angolo alto destro
FILLCOLOR [50, 50, 50]		; fissa un grigio più scuro per i 
                                ; cerchietti
FORWARD A
PENDOWN
CIRCLE 5
PENUP
FORWARD 10 LEFT 90 FORWARD 15
LABEL "[PAGESIZE[0], 0]"
BACK 15 RIGHT 90 BACK 10

RIGHT 90			; angolo basso destro
FORWARD B
PENDOWN
CIRCLE 5
PENUP
FORWARD 15 LEFT 90 FORWARD 30 LEFT 90
LABEL "[PAGESIZE[0],PAGESIZE[1]]"
LEFT 90 FORWARD 30 RIGHT 90 FORWARD 15
RIGHT 180

RIGHT 90			; angolo basso sinistro
FORWARD A
PENDOWN
CIRCLE 5
RIGHT 90
PENUP
BACK 15
LABEL "[0, PAGESIZE[1]]"
FORWARD 15

FORWARD B			; angolo altro sinistro
PENDOWN
CIRCLE 5
PENUP
FORWARD 15
LEFT 90 FORWARD 10 RIGHT 90
LABEL "[0, 0]"
BACK	 15
PENDOWN

HIDETURTLE			; mando la tartaruga a dormire

\end{lstlisting}

\vskip 1cm


%https://en.wikibooks.org/wiki/LaTeX/Source_Code_Listings
\lstset{literate=
  {á}{{\'a}}1 {é}{{\'e}}1 {í}{{\'i}}1 {ó}{{\'o}}1 {ú}{{\'u}}1
  {Á}{{\'A}}1 {É}{{\'E}}1 {Í}{{\'I}}1 {Ó}{{\'O}}1 {Ú}{{\'U}}1
  {à}{{\`a}}1 {è}{{\`e}}1 {ì}{{\`i}}1 {ò}{{\`o}}1 {ù}{{\`u}}1
  {À}{{\`A}}1 {È}{{\'E}}1 {Ì}{{\`I}}1 {Ò}{{\`O}}1 {Ù}{{\`U}}1
  {ä}{{\"a}}1 {ë}{{\"e}}1 {ï}{{\"i}}1 {ö}{{\"o}}1 {ü}{{\"u}}1
  {Ä}{{\"A}}1 {Ë}{{\"E}}1 {Ï}{{\"I}}1 {Ö}{{\"O}}1 {Ü}{{\"U}}1
  {â}{{\^a}}1 {ê}{{\^e}}1 {î}{{\^i}}1 {ô}{{\^o}}1 {û}{{\^u}}1
  {Â}{{\^A}}1 {Ê}{{\^E}}1 {Î}{{\^I}}1 {Ô}{{\^O}}1 {Û}{{\^U}}1
  {œ}{{\oe}}1 {Œ}{{\OE}}1 {æ}{{\ae}}1 {Æ}{{\AE}}1 {ß}{{\ss}}1
  {ű}{{\H{u}}}1 {Ű}{{\H{U}}}1 {ő}{{\H{o}}}1 {Ő}{{\H{O}}}1
  {ç}{{\c c}}1 {Ç}{{\c C}}1 {ø}{{\o}}1 {å}{{\r a}}1 {Å}{{\r A}}1
  {€}{{\euro}}1 {£}{{\pounds}}1 {«}{{\guillemotleft}}1
  {»}{{\guillemotright}}1 {ñ}{{\~n}}1 {Ñ}{{\~N}}1 {¿}{{?`}}1
}
\lstset{extendedchars=true, basicstyle=\scriptsize} 
\begin{lstlisting}[frame=single]  % Start your code-block

;  FIGURA 2 (Torna alla figura)
; L'effetto dell'istruzione POSITION e le coordinate della pagina
; Versione 1.
; A.R. Formiconi
; 25 luglio 2016


; Versione scritta come viene, senza particolari ottimizzazioni

;  Predispone tutto

CLEARSCREEN
HOME
A = 200				; lato corto del rettangolo che 
                                ; rappresenta la pagina
B = 282				; lato lungo
FILLCOLOR [230, 230, 230]	; fissa a grigio chiaro il colore di 
                                ; riempimento
PENUP
FORWARD B/2			; si dirige verso l'angolo in alto 
                                ; a sinistra
HEADING 9h
FORWARD A/2
HEADING 3h			; e punta verso destra (ore 3)
PENDOWN

; disegna il rettangolo, ruotando in senso orario

FORWARD A
RIGHT 90
FORWARD B
RIGHT  90
FORWARD A
FILL				; riempie il rettangolo
PENUP
RIGHT 90
FORWARD B			; si posiziona nell'angolo in alto 
                                ; a sinistra del foglio

; ripete il giro marcando gli angoli e ponendo le scritte 

HEADING 3h			; angolo alto destro
FILLCOLOR [50, 50, 50]		; fissa un grigio più scuro per i 
                                ; cerchietti
FORWARD A
PENDOWN
CIRCLE 5
PENUP
FORWARD 10 LEFT 90 FORWARD 15
LABEL "[PAGESIZE[0], 0]"
BACK 15 RIGHT 90 BACK 10

RIGHT 90			; angolo basso destro
FORWARD B
PENDOWN
CIRCLE 5
PENUP
FORWARD 15 LEFT 90 FORWARD 30 LEFT 90
LABEL "[PAGESIZE[0],PAGESIZE[1]]"
LEFT 90 FORWARD 30 RIGHT 90 FORWARD 15
RIGHT 180

RIGHT 90				; angolo basso sinistro
FORWARD A
PENDOWN
CIRCLE 5
RIGHT 90
PENUP
BACK 15
LABEL "[0, PAGESIZE[1]]"
FORWARD 15

FORWARD B				; angolo altro sinistro
PENDOWN
CIRCLE 5
PENUP
FORWARD 15
LEFT 90 FORWARD 10 RIGHT 90
LABEL "[0, 0]"
BACK	 15
PENDOWN

; disegna il percorso della tartaruga in seguito all'istruzione 
; \POSITION [350,320]

HOME					; mi posiziono al centro
PENUP					; senza disegnare...
POSITION [298, 430]			; vado dove voglio piazzare...
HEADING 0				; l'etichetta con le...
LABEL "[298, 421]"			; coordinate del centro
HOME					; torno al centro e...
PENDOWN					; disegno...
CIRCLE 5				; il cerchietto centrale
FILLCOLOR "green"			; fisso il colore verde per la 
                                        ; tartaruga, che mostrerò
POSITION [350,320]			; applico POSITION [350,320]
P = POSITION				; memorizzo tale posizione...
H = HEADING				; e direzione
PENUP					; alzo la penna
POSITION [300,320]			; mi sposto un po' per piazzare...
HEADING 0				; orientata correttamente
LABEL "[350, 320]"			; l'etichetta
POSITION P				; ritorno alla posizione e 
                                        ; direzione...
HEADING H				; per lasciarvi visibile la 
                                        ; tartaruga
\end{lstlisting}

\vskip 1cm

%https://en.wikibooks.org/wiki/LaTeX/Source_Code_Listings
\lstset{literate=
  {á}{{\'a}}1 {é}{{\'e}}1 {í}{{\'i}}1 {ó}{{\'o}}1 {ú}{{\'u}}1
  {Á}{{\'A}}1 {É}{{\'E}}1 {Í}{{\'I}}1 {Ó}{{\'O}}1 {Ú}{{\'U}}1
  {à}{{\`a}}1 {è}{{\`e}}1 {ì}{{\`i}}1 {ò}{{\`o}}1 {ù}{{\`u}}1
  {À}{{\`A}}1 {È}{{\'E}}1 {Ì}{{\`I}}1 {Ò}{{\`O}}1 {Ù}{{\`U}}1
  {ä}{{\"a}}1 {ë}{{\"e}}1 {ï}{{\"i}}1 {ö}{{\"o}}1 {ü}{{\"u}}1
  {Ä}{{\"A}}1 {Ë}{{\"E}}1 {Ï}{{\"I}}1 {Ö}{{\"O}}1 {Ü}{{\"U}}1
  {â}{{\^a}}1 {ê}{{\^e}}1 {î}{{\^i}}1 {ô}{{\^o}}1 {û}{{\^u}}1
  {Â}{{\^A}}1 {Ê}{{\^E}}1 {Î}{{\^I}}1 {Ô}{{\^O}}1 {Û}{{\^U}}1
  {œ}{{\oe}}1 {Œ}{{\OE}}1 {æ}{{\ae}}1 {Æ}{{\AE}}1 {ß}{{\ss}}1
  {ű}{{\H{u}}}1 {Ű}{{\H{U}}}1 {ő}{{\H{o}}}1 {Ő}{{\H{O}}}1
  {ç}{{\c c}}1 {Ç}{{\c C}}1 {ø}{{\o}}1 {å}{{\r a}}1 {Å}{{\r A}}1
  {€}{{\euro}}1 {£}{{\pounds}}1 {«}{{\guillemotleft}}1
  {»}{{\guillemotright}}1 {ñ}{{\~n}}1 {Ñ}{{\~N}}1 {¿}{{?`}}1
}
\lstset{extendedchars=true, basicstyle=\scriptsize} 
\begin{lstlisting}[frame=single]  % Start your code-block

;  FIGURA 3 (Torna alla figura)
; I riferimenti per l'istruzione HEADING
; Versione 1.
; A.R. Formiconi
; 25 luglio 2016


; Versione scritta in maniera più ordinata e strutturata, facendo 
; uso delle "procedure". La strutturazione del software di questo 
; tipo consente di rendere il codice più modulare e più mantenibile.

; Prima vengono le subroutine e alla fine il programma vero e 
; proprio. Prima vanno messi i frammenti più elementari, in maniera 
; che LibreLogo li legga per primi. Questo serve perché quando nel 
; codice viene citata una subroutine, questa deve essere già stata 
; analizzata da LibreLogo.


; Subroutine BR per disegnare un segmento lungo 10 pt dalla 
; posizione e lungo la  direzione corrente senza muoversi 
; (come risultato finale)
 
; Parametri:
; 	P: posizione corrente
;	H: direzione corrente

TO BR P H
	FORWARD 10
	POSITION P
	HEADING H
END

; Subroutine TARROW per disegnare la punta di una freccia

TO TARROW
	P = POSITION
	H = HEADING	
	LEFT 160
	BR P H	
	RIGHT 160
	BR P H
END

; Subroutine LBA scrivere il testo contenuto in T in vetta al segmento
; di lungheza L e ruotato dell'angolo A
; Parametri:
; 	L: lunghezza del segmento
;	A: angolo di rotazione del segmento
;	T: testo da scrivere nell'etichetta

TO LBA L A T
	PENUP
	FORWARD L/10 + L/5 * SIN A*PI/180
	H = HEADING 
	HEADING 0
	LABEL T
	PENDOWN
END

; Subroutine LB scrivere il testo contenuto in T in una posizione 
; determinata in coordinate polari rispetto alla posizione corrente, 
; mediante la  distanza L l'angolo A.

; Parametri:
; 	L: distanza
;	A: angolo 
;	T: testo da scrivere nell'etichetta

TO LB L A T
	P0 = POSITION
	PENUP
	S =  SIN A*PI/180
	C =  COS A*PI/180
	POSITION [P0[0] + L * C, P0[1] - L * S]
	HEADING 0
	LABEL T
	POSITION P0
	HEADING 0
	PENDOWN
END

; Subroutine ARROW per disegnare, a partire dalla posizione corrente, 
; una freccia di lunghezza L, inclinata di un angolo A, e in vetta una
; etichetta con il testo contenuto in T
; Parametri:
; 	L: lunghezza della freccia
;	A: angolo di rotazione della freccia
;	T: testo da scrivere nell'etichetta

TO ARROW P0 A0 L A T
	PENDOWN
	HEADING A
	FORWARD L
	TARROW
	LBA L A T
	PENUP
	POSITION P0
	HEADING A0
	PENDOWN
END

; Questo è il programma vero e proprio che, come si vede, grazie al 
; ricorso alle procedure, è abbastanza conciso.

CLEARSCREEN			; cancello il foglio
HOME				; mando a casa la tartaruga
HIDETURTLE			; faccio il disegno senza vedere i
                                ; la tartaruga
FILLCOLOR [230, 230, 230]	; fisso il riempimento a un grigio scuro
CIRCLE 5			; disegno un cerchietto nella posizione 
                                ; centrale 
P0 = POSITION			; memorizzo tale posizione iniziale...
A0 = HEADING			; e anche la direzione iniziale
L = 150				; faccio la freccia lunga 150 pt
A = 60				; e la voglio inclinata di 60 gradi

PENSIZE 1 ARROW P0 A0 L 0 "HEADING 0"	  ; freccia verticale
PENSIZE 0.5 ARROW P0 A0 L A "HEADING 30"  ; freccia a 60 gradi
PENSIZE 1 ARROW P0 A0 L 90 "HEADING 90"	  ; freccia orizzontale
PENSIZE 0.5	
ELLIPSE [L/3, L/3, 0, A, 3]	; arco di cerchio piccolo
LB L/4 A "30 gradi" 		; etichetta "30 gradi"
ELLIPSE [L*2, L*2, 0, 90, 3] 	; arco di cerchio grande

\end{lstlisting}

\vskip 1cm


%https://en.wikibooks.org/wiki/LaTeX/Source_Code_Listings
\lstset{literate=
  {á}{{\'a}}1 {é}{{\'e}}1 {í}{{\'i}}1 {ó}{{\'o}}1 {ú}{{\'u}}1
  {Á}{{\'A}}1 {É}{{\'E}}1 {Í}{{\'I}}1 {Ó}{{\'O}}1 {Ú}{{\'U}}1
  {à}{{\`a}}1 {è}{{\`e}}1 {ì}{{\`i}}1 {ò}{{\`o}}1 {ù}{{\`u}}1
  {À}{{\`A}}1 {È}{{\'E}}1 {Ì}{{\`I}}1 {Ò}{{\`O}}1 {Ù}{{\`U}}1
  {ä}{{\"a}}1 {ë}{{\"e}}1 {ï}{{\"i}}1 {ö}{{\"o}}1 {ü}{{\"u}}1
  {Ä}{{\"A}}1 {Ë}{{\"E}}1 {Ï}{{\"I}}1 {Ö}{{\"O}}1 {Ü}{{\"U}}1
  {â}{{\^a}}1 {ê}{{\^e}}1 {î}{{\^i}}1 {ô}{{\^o}}1 {û}{{\^u}}1
  {Â}{{\^A}}1 {Ê}{{\^E}}1 {Î}{{\^I}}1 {Ô}{{\^O}}1 {Û}{{\^U}}1
  {œ}{{\oe}}1 {Œ}{{\OE}}1 {æ}{{\ae}}1 {Æ}{{\AE}}1 {ß}{{\ss}}1
  {ű}{{\H{u}}}1 {Ű}{{\H{U}}}1 {ő}{{\H{o}}}1 {Ő}{{\H{O}}}1
  {ç}{{\c c}}1 {Ç}{{\c C}}1 {ø}{{\o}}1 {å}{{\r a}}1 {Å}{{\r A}}1
  {€}{{\euro}}1 {£}{{\pounds}}1 {«}{{\guillemotleft}}1
  {»}{{\guillemotright}}1 {ñ}{{\~n}}1 {Ñ}{{\~N}}1 {¿}{{?`}}1
}
\lstset{extendedchars=true, basicstyle=\scriptsize} 
\begin{lstlisting}[frame=single]  % Start your code-block

;  FIGURA 4 (Torna alla figura)
; Esempi di spessore del tratto
; Versione 1.
; A.R. Formiconi
; 25 luglio 2016

; In questo esempio si illustra l'impiego dell'istruzione REPEAT, 
; per realizzare i "cicli" ("loop")

CLEARSCREEN
HOME
RIGHT 90
PENCOLOR "BLACK"

REPEAT  10 [
	PENWIDTH REPCOUNT
	FORWARD 100
	PENUP FORWARD 50
	HEADING 0h
	LABEL  "PENWIDTH " + STR REPCOUNT-1 
	HEADING 3h
	BACK 50
	PENUP
	RIGHT 90
	FORWARD 20
	RIGHT 90
	FORWARD 100
	LEFT 180
	PENDOWN
]
PENWIDTH 0
HIDETURTLE

\end{lstlisting}

\vskip 1cm


