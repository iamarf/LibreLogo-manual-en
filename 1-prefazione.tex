\part{Basics of Logo and LibreLogo} \label{parte:manuale}

\section{Preface}

This is an English version derived from the “Piccolo Manuale di LibreLogo” that
I wrote for the students of the Primary Education curriculum at the university
of Florence. The manual was intended to aid the students in developing their
coding activities within the "Laboratorio di Tecnologie Didattiche" (Laboratory
of Educational Technologies). The programming language is LibreLogo, which is
an implementation of Seymour Papert's Logo language within the Writer word
processor of the LibreOffice  suite. LibreLogo is a plugin available by default
in Writer since versione 4.0 of LibreOffice. It has been written in Python by
László Németh. The specific documentation can be found at http://librelogo.org.
From there a description of all the LibreLogo commands can be downloaded in 33
different languages \footnote{The commands dictionaries can be downloaded from
https://help.libreoffice.org/Writer/LibreLogo\_Toolbar}. Apart from this, some
extended documentations can be found only in Hungarian, as far as I know as of
March 2018: there are a pretty technical handbook written by László Németh
itself \cite{LibreLogo2} and a more classroom oriented one by Lakó Viktória
\cite{LibreLogo3}. Some of the examples presented in this work have been
inspired by those in  Lakó Viktória's book,  but the perpespective is different
here. First of all I have put  a strong emphasis on pedagogical aspects, beyond
the mere technical facts, following the line of thought of Seymour Papert, as
reported, for instance, in \textit{Mindstorms} \cite{Papert}. Secondly, during
the couple of years since the writing of the first version of the “Piccolo
Manuale di LibreLogo”,  a number of reflections, exercises and hands-on
practices got back from students and other sources, have been added. The
overall result is a rather broad discussion of possible uses of LibreLogo, both
in a vertical dimension, from primary school examples to tertiary education
level exercises, as well as in a transversal dimension, through a variety of
disciplines. Finally, this is not a true translation of the “Piccolo Manuale di
LibreLogo” but an English and somewhat more concise rewriting.


