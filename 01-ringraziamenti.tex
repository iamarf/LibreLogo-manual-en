\tableofcontents
\newpage

\section{Aknowldegments}

First af all I thank all the students who worked beyond my expectations, even
by sending works that they were not supposed to do. Their explosion of
creativity was extremely useful. Some of these contributes deserve specific
mentions. First the remarkable "Creativity Exercises" of Marta Veloce, Primary
Education student, that inspired chapter \ref{cap:marta}. Then Alberto Averono,
teacher of a secondary school, who in a training course about coding further
developed Marta's exploration. Another Primary Education student, Eleonora
Aiazzi, sent a moving text, "When a professor treats you as a child", who was
extremely encouraging about the teaching method I was experimenting. A well
experienced primary school teacher, Antonella Colombo, gave me back a wonderful
documentation of Papert's syntonic learning she tried to trigger in her
classroom. Her work opens the trip in chapter \ref{cap:cerchio}, from the draw
of a circle to the Halley's orbit. Thanks to Piero Salonia, who corrected the
drafts of the first Italian version of this manual. Finally, thanks to the
beautiful world of Linux and its sharp tools  - scp, ssh, rsync, grep, find,
nmap, latex, bibtex and so on - which gives true superpowers, unknown in the
Graphical User Interface World - the Internet of true freedom.


