\part{Manuale ragionato di LibreLogo}

\section{Prefazione}
Questo piccolo manuale nasce per la necessità di fornire supporto di studio e
consultazione nell'insegnamento “Laboratorio di Tecnologie Didattiche” al V
anno del Corso di Laurea Magistrale a ciclo unico “Scienze della Formazione
Primaria” e nell'insegnamento “Laboratorio di Gestione dei Processi Formativi”
al II anno del Corso di Laurea Magistrale “Scienze dell'Educazione degli
Adulti, della Formazione Continua e Scienze Pedagogiche”, presso l'Università
di Firenze, e nell'insegnamento “Informatica” al I anno del Corso di Laurea
Magistrale “Innovazione Educativa e Apprendimento Permanente” presso
l'università telematica Italian University Line. Il manuale guida all'impiego
del linguaggio Logo nella versione LibreLogo implementata all'interno del \textit{word}
\textit{processor} Writer della \textit{suite} di programmi di produttività personale
LibreOffice. LibreLogo è un \textit{plugin} disponibile di \textit{default} in Writer a partire
dalla versione 4.0 di LibreOffice. È stato scritto in linguaggio Python da
László Németh. La documentazione disponibile si trova in http://librelogo.org,
da dove, in particolare, si può scaricare una guida dei comandi di LibreLogo in
italiano \cite{LibreLogo}. Per il resto, sfortunatamente e per quanto è a mia conoscenza sino ad oggi, la documentazione disponibile è tutta in ungherese, principalmente sotto forma di un manuale di esempi scritto dallo stesso László Németh \cite{LibreLogo2} e da un manuale esteso scritto da Lakó Viktória \cite{LibreLogo3}. È a quest'ultimo lavoro che, in una prima fase si è ispirato il presente piccolo manuale, senza tuttavia esserne una traduzione, per vari motivi. In primo luogo io non so l'ungherese e non posso quindi pretendere di poterne fare una vera traduzione e i tempi e le circostanze non mi consentono di avvalermi di un traduttore. Posso tuttavia seguirne le tracce, aiutandomi con i codici (anche se in ungherese quelli si possono imparare), le figure e Google Translate. Del resto, alla fine una traduzione pedissequa non sarebbe nemmeno desiderabile perché viene naturale riformulare il materiale in funzione degli obiettivi specifici e della propria visione della materia. Inoltre, nel corso della traduzione, mi è capitato sempre più spesso di seguire la traccia dei miei pensieri e, alla fine, è stato inevitabile tornare alla fonte primigenia, ovvero al testo con cui Seymour Papert descrisse per la prima volta compiutamente il pensiero che aveva dato origine a Logo, \textit{Mindstorms} \cite{Papert}. È così che ho introdotto la traduzione di due capitoli di \textit{Mindstorms}: il secondo, “\textit{Mathofobia: the Fear of Learning}”, e il terzo, \textit{“Turtle Geometry: A Mathematics Made for Learning”}.

L'immersione profonda nel pensiero di Papert ha poi prodotto un fenomeno interessante. Nei
numerosi passaggi dove Papert insiste sulla necessità di proporre agli studenti nuove idee
matematiche facendo leva sulle conoscenze già possedute (non solo scolastiche) dagli studenti e sul loro coinvolgimento personale, sempre più spesso mi venivano in mente le lezioni di Emma
Castelnuovo, con le quali si impiegano materiali semplici per introdurre tanti concetti matematici. Ad esempio \cite{Castelnuovo}. In questo
libro si riportano alcune lezioni fatte da Emma Castelnuovo presso la Casa-laboratorio di Cenci (Franco Lorenzoni), fra il 2002 e il 2007. La ricerca didattica di Emma Castelnuovo ha riguardato molto l'impiego di materiali semplici per lo studio attivo della matematica.:

\begin{quote}
Ho capito, insomma, che partendo da un materiale semplicissimo (sbarrette, spaghi, elastici ecc.) si potevano costruire i vari capitoli della geometria, motivando i ragazzi a partire da problemi reali. Bastava variare qualche elemento, lasciandone invariati altri, per stimolare delle problematiche anche di alta matematica. Bastava saper guardare attorno a noi perché si aprissero nuove vie del pensiero e si arrivasse, quasi da sé, a formare negli allievi uno spirito matematico.
\end{quote}

Questo pensiero è in accordo completo con quello di Papert. L'unica differenza è costituita dal
contesto nel quale i due autori vanno a ricercare l'interesse e il coinvolgimento degli allievi. Si può dire che la geometria della Tartaruga è un analogo dei materiali fisici usati da Emma Castelnuovo.
Le due visioni e le pratiche che ne scaturiscono non sono affatto in opposizione bensì
complementari. In questa prospettiva, con LOGO si continua e si estende il lavoro
(necessariamente) iniziato con i materiali fisici mantenendo lo stesso identico approccio
pedagogico.

Tutte le figure sono state prodotte con LibreLogo stesso. I codici, adeguatamente commentati, di alcune delle figure sono listati in appendice, come esempio e spunto per ulteriori sviluppi. Nel momento in cui scrivo queste righe ho completato solo il primo capitolo ma trovo utile rendere il lavoro disponibile anche per ricevere eventuali riscontri che potrebbe essere utile per il resto.



